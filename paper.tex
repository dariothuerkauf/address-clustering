%%%%%%%%%%%%%%%%%%%%%%%%%%%%%%%%%%%%%%%%%%%%%%
%%%%%%%%%%%%%%%%%%%%%%%%%%%%%%%%%%%%%%%%%%%%%%
%%% Master Thesis Template by Fabian Schär %%%
%%%%%%%%%%%%%%%%%%%%%%%%%%%%%%%%%%%%%%%%%%%%%%
%%%%%%%%%%%%%%%%%%%%%%%%%%%%%%%%%%%%%%%%%%%%%%

%%%%%%%%%%%%%%%%%%%%%%%%%%%%%%%%%%%%%%
%%% Packages and Document Settings %%%
%%%%%%%%%%%%%%%%%%%%%%%%%%%%%%%%%%%%%%

\documentclass[12pt,a4paper,titlepage,oneside,english]{article}

%%% Main Packages %%%
\usepackage[english]{babel}
%\usepackage[ngerman]{babel} % Use this option for German settings.
\usepackage[T1]{fontenc}
\usepackage[utf8]{inputenc}

%%% Additional Packages %%%
\usepackage{cite}
\usepackage{framed}
\usepackage{graphicx}
%\usepackage[german]{fancyref}
\usepackage[german,hidelinks]{hyperref} %hidelinks
\usepackage{multirow}
\usepackage[round]{natbib}
\usepackage{setspace}
\usepackage{geometry}
\usepackage{pst-all} % Not working with Sweave!!!

%%% Math Packages %%%
\usepackage{amsmath}
\usepackage{amstext}
\usepackage{amssymb}
\usepackage{theorem}
\usepackage{epsfig}
\usepackage{longtable}

%%% Layout Specifications %%%
\geometry{a4paper, top=35mm, left=40mm, right=40mm, bottom=45mm,
headsep=10mm, footskip=12mm}

%%% Parskip Settings %%%
\setlength{\parskip}{3mm}
\setlength{\parindent}{0mm}

%%% Document Specifications %%%
\title{Template Master's Thesis}
\author{John Doe}


%%%%%%%%%%%%%%%%%%
%%% Title Page %%%
%%%%%%%%%%%%%%%%%%

\begin{document}
%\begin{titlepage}
\begin{center}
\vspace{1em}
\large{Master Thesis}\\
\huge Developing Address Clustering Heuristics for Account-Based Blockchain Networks:\\ An Analysis based on a Specific Address Set \\
\Large \vspace{1em}
Dario Thürkauf
\end{center}

\vspace{1em}
\normalsize
\begin{flushleft}
Supervised by:\\ 
Prof. Dr. Fabian Schär \\
Credit Suisse Asset Management (Schweiz) Professor for \\ 
Distributed Ledger Technologies and Fintech \\
Center for Innovative Finance, University of Basel
\end{flushleft}

\vspace{1em}
\onehalfspacing
\begin{center}
\section*{Abstract}
\end{center}
Assentiar consuetae ha opinionum mentemque ob ii. Ne conflantur de intelligat et me cohibendam. Imaginandi ob to at agnoscerem et mutationum. In methodum ob ii at quicquid lectorum. Procuravi ha dependent ob evidenter tangantur concipere. Immortalem objectivus deo eae rei attingebam ita advertebam quamprimum. Typis patet prius qua nia mem ens. Suppono sim ita pendere nam agnosci quopiam vestiri spondeo dum. Tes illum mundo vetus signa fit talem res his.  \\
\vfill
\textbf{Keywords:} Keyword 1, Keyword 2, Keyword 3, Keyword 4.\\
\noindent\textbf{JEL:} X00, X00, X00
%\end{titlepage}


%%%%%%%%%%%%%%%%%%%%%%%%%%%%%
%%% Contents & Declaration%%%
%%%%%%%%%%%%%%%%%%%%%%%%%%%%%

\pagenumbering{gobble}

\newpage
\pagenumbering{Roman}
\tableofcontents

\vfill
\begin{center}
\includegraphics[width=4cm]{../figures/logo_cif.png}
\end{center}
\newpage
\singlespacing
%\vspace{-1.5cm}
\section*{Declaration of Independent Authorship}
I attest with my signature that I have written this work independently and without outside help. I also attest that the information concerning the sources used in this work is true and complete in every respect. All sources that have been quoted or paraphrased have been marked accordingly. 
Additionally, I affirm that any text passages written with the help of AI-supported technology are marked as such, including a reference to the AI-supported program used. This paper may be checked for plagiarism and use of AI-supported technology using the appropriate software. I understand that unethical conduct may lead to a grade of 1 or ``fail`` or expulsion from the study program.\\

Dario Thürkauf

\begin{figure}[h!]
	\centering
	\hspace{-10cm}
	\includegraphics[width=3cm]{../figures/signature.jpeg}
\end{figure}
%%%%%%%%%%%%%%%%%%%%

\newpage
\onehalfspacing
\pagenumbering{arabic}


%%%%%%%%%%%%%%%%%%%
%%% Introduction%%%
%%%%%%%%%%%%%%%%%%%

\section{Introduction}

Public permissionless blockchains such as Bitcoin \citep{nakamotoBitcoin2008} and Ethereum \citep{buterin2014ethereum} allow users %individuals
 to participate with multiple pseudonymous addresses. The creation of these addresses is virtually cost free. Contrary to popular belief, these blockchains are entirely transparent. All transactions are stored as part of the blockchain's history and publicly observable.
This has opened up a nascent scientific field dealing with entity identification within blockchain networks. Researchers try to cluster addresses controlled by the same user by analyzing on-chain data and detecting usage patterns. The frequent lack of ground truth makes it difficult to evaluate different clustering methods. As a result, most methods are heuristic. %The majority of these methods are heuristic which often makes them difficult to evaluate due to the absence of any ground truth.
\newline On one hand, identifying the addresses that belong to the same real-world entity is beneficial. According to \cite{FV:17} it allows better evaluation of network properties with respect to usage, distribution of wealth and detecting fraudulent activities. For instance, if a user distributes their voting power across various addresses, they might manipulate an on-chain voting process that seems fair at the outset. \newline
Conversely, the lack of privacy is also detrimental to most financial use cases (DeFi). As a response, multiple privacy-enhancing tools and protocols have been proposed to obfuscate transaction tracing. 
Nonetheless, these protocols are not yet widely adopted and careless user behaviour can undermine the privacy guarantees they offer. \newline
%As such/In conclusion
The topic of privacy, or lack thereof, will remain significant for future blockchain development and research.

%The most prominent/widely used non-custodial mixer is Tornado cash (Schär, Tutela).
%If someone obtains information that allows them to link a blockchain address to an entity, they may effectively observe that entity’s entire transaction history and associated activity. Even if the entity uses multiple addresses, any link between these addresses may expose the fact that they belong to the same person.

\textbf{Related work}\\
Previous work in this field can be broadly distinguished between methods %heuristics
 developed for the unspent transaction output (UTXO) model (e.g, Bitcoin and ZCash) and the account-based model (e.g., Ethereum, Polygon). While both models share the concept of addresses, the notion of accounts is not present in UTXO-based blockchains. The way in which transactions are processed is fundamentally different, and therefore, clustering heuristics are not applicable to both paradigms.

A number of address clustering heuristics in the UTXO setup have been proposed for Bitcoin and derivatives \citep{Androulaki2013, Meiklejohn2013, Haslhofer2016, jourdan2018, kappos2022}. All these methods will not be the focus of this work and therefore not discussed. As suggested by \cite{nakamotoBitcoin2008} in the Bitcoin whitepaper, most Bitcoin wallet implementations use a new key pair for each transaction to keep them from being linked to a common owner. \newline
In contrast to the UTXO-model, native transactions in account-based blockchains can only move funds between a single sender and receiver, and the ``change`` remains in the sender account. Subsequent transactions necessarily use the same address again. The account-based model essentially relies on address reuse on the protocol level \citep{Beres2020}. Therefore, privacy guarantees should be lower and most users likely use only a small number of addresses. \newline
% Make an assumption that people only use a small set of addresses
Clustering heuristics for account-based blockchains were first introduced by \cite{FV:17}. He proposed and applied heuristics that exploit patterns related to deposit addresses, multiple participation in airdrops, and token authorization mechanisms. \newline
\cite{Beres2020} propose more universally applicable methods, as they argue that Victor’s heuristics, while powerful, assume participation in certain on-chain events. Their technique interprets transactions or token transfers as network graphs, with addresses as nodes and asset transfers as edges. They quantitavely compare graph-representation learning algorithms (a subset of machine learning) and propose further user profiling techniques based on time-of-day activity and transaction fees. Using ENS address pairs as ground truth, the authors rigorously test their methods and apply their findings to significantly reduce the privacy guarantees of Tornado Cash, a non-custodial privacy-enhancing protocol on Ethereum. \newline
\cite{wu2022tutela} extend on one of \cite{Beres2020} graph representation learning %node-embedding
 algorithms and apply it at scale. Further, they also propose a set of new clustering heuristics targeting Tornado Cash,  showing that careless user behavior can still reveal identity. They build an application built on those heuristics to measure the anonymity of an Ethereum address. All of the methods mentioned will be discussed in greater detail in Section 3. \newline
Broadening the scope of entity identification, \cite{victorlüders2019} study the largest Ethereum ERC-20 token networks from a graph perspective. Similarly, \cite{casalebrunet2021} apply network graph analysis to various non-fungible token (NFT) ecosystems. Both find that many of the networks follow either a star or a hub-and-spoke pattern. These patterns are common to interaction graphs in social networks. Further, \cite{Payette2017} propose a segmentation of the Ethereum address space into four distinct behaviour groups sharing similar attributes using k-means clustering. \newline
Rather than treating users as entities and analyzing on-chain data, \cite{yu2023} propose a novel approach for correlation analysis by exploiting network information. Although this approach has the potential to avoid the impact of privacy-enhancing technologies, it introduces a new set of limitations and problems. Thus, this approach may be of great interest for further research, particularly when privacy-enhancing techniques become more widely used.


\textbf{Our contribution}\\
In this work, we perform entity identification on an address set containing 473,927 addresses. To obtain these addresses, a separate project collected snapshots of avatar activity in Decentraland over an ten-month period from July 2022 to April 2023. Due to Decentraland's blockchain-based architecture, each avatar contains information about a user's Web3 address. \newline
We test the applicability of existing heuristics, and if feasible, their efficacy in detecting entities using multiple addresses is evaluated.
The main objective is to estimate the number of real-world entities that these addresses represent. The clustering of addresses is limited to this set only, without considering any %(clusters for)
 addresses outside of it, e.g., that were not registered in Decentraland during the given time frame.
Furthermore, we introduce our own heuristics for identifying address clusters and evaluate their suitability for implementation in other contexts.
%Anwenden/discuss and evaluate existing heuristics to my specific address set.
%Anpassen, erweitern

The remainder of this paper is structured as follows: In Section 2, we provide a brief overview of the basic concepts necessary to understand the setup. The include Ethereum accounts and the EVM, tokens, Decentraland, and privacy-enhancing protocols. Section 3 describes our methodology for data collection and preparation. The fourth section provides a detailed explanation of various clustering heuristics.The fifth section applies the clustering heuristics described in section 4, after which we discuss the results of our analysis and summarize our findings in sections 6 and 7.

%%%%%%%%%%%%%%%%%%%%%
%%% Preliminaries %%%
%%%%%%%%%%%%%%%%%%%%%

\section{Preliminaries}
% Introduce section
\subsection{Contract Accounts and EOAs}
Account-based blockchains usually distinguish between \textit{externally owned accounts} (EOAs) and \textit{contract accounts} (smart contracts). EOAs are created and controlled by private keys and can be used to hold the native protocol asset, send and receive transactions, and interact with contract accounts. Contract accounts are controlled by the contract's code, their state can be modified through transactions sent to the contract and they cannot initiate transactions. \citep{buterin2014ethereum} \newline Each account has a 20-byte address encoded in hexadecimal. For EOAs, this address is based on the last 20 bytes of the Keccak-256 hash of the ECDSA public key. For contract accounts, it is usually the last 20 bytes of the Keccak-256 hash of the RLP encoding of the sender address and account nonce. \citep{GW:14}

\subsection{Ethereum, Polygon and the EVM}
Ethereum and Polygon are both smart contract based account-model blockchains. They share the same execution logic - the Ethereum Virtual machine (EVM).
The EVM provides a standardized framework for contract execution, ensuring that smart contracts produce deterministic results across all nodes in the network \citep{GW:14}. The EVM is independent of the underlying blockchain protocol. This allows other blockchain implementations to adopt this standardized framework for contract execution.\newline
Polygon is a so-called sidechain and aims to address the scalability limitations of Ethereum. The adoption of the EVM means that both blockchains share the same user address schemes. It further enables developers to deploy and execute Ethereum-based smart contracts on the Polygon network with minimal changes. \citep{matic_whitepaper}
%A high-level programming language is typically used to write smart contracts, and a compiler to convert them into bytecode \citep{mastering_ethereum}. Solidity is currently the dominant smart contract language.
%Polygon PoS is a so-called sidechain, closely associated with Ethereum mainnet and used for scaling.
%Ethereum Mainnet is the chain on which much of the Decentraland infrastructure has been deployed.
%In addition, \cite{eip1014} introduces a new contract creation mechanism allowing for more predictable contract addresses.%EIP-1014
%This allows developers to calculate the address of a contract before it is deployed.


\subsection{Tokens and Token Standards}
\cite{roth2019tokenization} define \textit{tokens} as rivalrous, digital units of value that represent ownership of an asset or utility. Smart contracts are the primary method of creating tokens on EVM-based blockchains. In essence, a smart-contract-based token is a mapping of accounts with token balances and a set of functions defining how these balances can be changed. Any smart contract containing these elements can be interpreted as a token contract. \citep{roth2019tokenization} \newline
Token standards specify basic interfaces that allow for interoperability between contracts. These standards do not prescribe an implementation but set minimum requirements without restricting the design beyond that \citep{mastering_ethereum}. 
The most common token standards are ERC-20\footnote{https://eips.ethereum.org/EIPS/eip-20} for fungible tokens, ERC-721\footnote{https://eips.ethereum.org/EIPS/eip-721} for non-fungible tokens (NFTs) and ERC-1155\footnote{https://eips.ethereum.org/EIPS/eip-1155} for semi-fungible tokens.\newline
When a (blockchain) transaction completes, it produces a transaction receipt that contains log entries providing information about the actions that occurred during its execution \citep{mastering_ethereum}. The standarization of tokens allows us to listen for ``transfer events``, which are emitted whenever an token changes ownership.
These transfer events include the \textit{sender} and \textit{recipient} address, along with a \textit{token ID} and/or \textit{amount}. 

%Token standards specify basic interfaces that allow for interoperability between contracts. These standards do not prescribe an implementation but set minimum requirements without restricting the design beyond that. ERC-20 (fungible), ERC-721 (non-fungible) or ERC-1155 (semi-fungible) are the most widely used token standards. \newline

%Solidity high-level objects called \textit{Events} construct these logs, which can be queried from a full node and are stored separately from the state.

%Token Definition
%Implemented in a standardized way
%Three most popular token standards
%Transfer emits Events/Logs/Transaction receipts, we can listen for Events

\subsection{Decentraland}
\textit{Decentraland} is the first large-scale blockchain-based virtual world. It has attracted the attention of well-known companies like Tommy Hilfiger, Samsung, PepsiCo, Diesel, Adidas, and Netflix, who are actively engaged in it  \citep{metaverse-retailing2023}.\newline
Decentraland is an ideal platform for empirical research for two main reasons.
\textit{First}, its open architecture allows compiling snapshots of user activity, including the users' locations  within the metaverse. This was done by \cite{metaverse-retailing2023}, who captured snapshots of users/avatar locations over a period of 10 months. The addresses used in this paper are provided form this work.
\textit{Second}, users who connect their Web3 account can own various digital assets, such as monetary units, land parcels, and avatar collectibles (e.g. wearables, emotes, and names). Smart contracts on Ethereum and Polygon track and manage the ownership of these assets \cite{goldbergschaer2023}. As a result, this allows us to analyze the entire transaction history and derive information about the avatars’/users' economic activity.  \newline 
The Decentraland ecosystem has two main tokens. \textit{LAND}, an NFT compliant with ERC-721, manages the ownership of land parcels in Decentraland. \textit{MANA}, a fungible token compliant with ERC-20, is the in-world currency used to purchase digital goods and services. The majority of purchases and trades are settled in MANA. Furthermore, Decentraland %the foundation/DAO/developers
operates/provides/hosts a virtual/smart-contract-based marketplace where its users may buy or exchange LAND or other in-game collectibles.
 
%Owners of multiple adjacent land parcels can combine them into an ESTATE, which acts as a wrapper token for the land parcels.
%The term metaverse refers to Neal Stephenson’s 1992 novel Snow Crash and describes an immersive virtual world that is populated by humans-as-avatars.
%Goods: in-world assets, such as avatar wearables, merchandise, or tickets, as well as tokenized vouchers that can be redeemed for real-world goods or services.

\subsection{Non-custodial crypto asset mixers}
%intuition of how it works,  distinction between custodial and non-custodial, Tornado cash = SC based
According to \cite{nadler2023tornado}, crypto-asset mixers are currently the most widely used approach to achieve privacy on public blockchains. At a high level, we can distinguish between custodial and non-custodial crypto-asset mixers. \newline
In the custodial setup, users send their assets to a centralized service provider's public deposit address. In return, the provider sends the assets back to a privately relayed recipient address. This approach is fully trust-based, as the service provider has complete control over the assets and access to the identifying data. \newline 
In contrast, non-custodial crypto asset mixers (e.g., Tornado Cash) replace the trusted mixing party with a publicly verifiable smart contract. They rely on cryptographic schemes that allow anyone to prove and independently verify the validity of a withdrawal without disclosing the link to a specific deposit. This has the advantage that users do not have to share the identifying information with anyone and there is no liquidity risk as the funds are locked. \citep{nadler2023tornado} \newline
In a nutshell, the mixer works in the following way: Various entities deposit the same amount of a specific crpyto asset to a mixer address acting as a pool. Anyone who has contributed to the pool may then generate a new address and withdraw their funds without revealing the link between the deposit and withdrawal addresses. This is achieved using zK-SNARKS. 
Each depositor inserts a hash value in a Merkle-tree. At the time of withdrawal, each legitimate withdrawer can prove unlinkably with a zero-knowledge proof that they know the pre-image of a previously inserted hash leaf in the Merkle-tree. %Subsequently, users can withdraw their asset from the mixer whenever they consider that the size of the anonymity set is satisfactory.
 For a more detailed description of how Tornado Cash works, see \cite{nadler2023tornado} or \cite{Beres2020}. For a mathematical discussion of ZKSNARKs, please refer to Petkus (2019), Thaler (2022) or Berentsen et al. (2022). \newline
In the context of this work, it is sufficient to know that third parties can still observe the addresses that have deposited to and withdrawn from the pool. The number of addresses that have deposited in a particular pool is defined as the \textit{anonymity set}. Typically, a larger anonymity set will provide a higher level of privacy, as a more deposit addresses can be associated with any given withdrawal address. \newline
In theory, if the anonymity set is large enough, third parties will not be able to link a specific depositor address to a specific withdrawal address. However, this is not always the case. For example, some users withdraw from the pool with the same address as they deposited. According to \cite{nadler2023tornado}, careless use of the protocol, testing transactions or external incentive mechanisms might be reasons for this behaviour. \newline
Furthermore, \cite{Beres2020} and \cite{wu2022tutela} have proposed approaches to de-anonymize Tornado cash transactions using various clustering heuristics. These methods are discussed in more detail in the next section.
% making use of address re-use, transaction activity, transaction cost choice, or more complex transaction graph and network analyses.

%The best known smart-contract based non-custodial mixer is Tornado Cash.
 

%%%%%%%%%%%%
%%% Data %%%
%%%%%%%%%%%%

\section{Data Collection and Preparation}
% Introduction Satz - Ausgangslage 
The starting point for our work was a dataset comprising 473,927 distinct (Web3) addresses. %The original dataset comprised 473,927 distinct (Web3) addresses.
 As we focus on clustering methods using on-chain data, we exclude addresses that have not been recorded on either Ethereum or Polygon. To accomplish this, we used data from Blockscan\footnote{\url{https://blockscan.com/}}. Overall, 59,651 addresses were recorded on Ethereum, 129,988 on Polygon, with 52,095 addresses appearing on both networks simultaneously. The set of remaining addresses consists of 137,544 addresses. Subsets are visualized as a Venn diagram in \ref{fig:Venn}.
 
\begin{figure}[h!]
	\centering
	\includegraphics[width=0.6\textwidth]{../figures/venn_diagram.png}
	\caption{Venn Diagram of address subsets}
	\label{fig:Venn}
\end{figure} 

\subsection{Transactions and Token Transfer Events}
The address dataset was used to collect transaction and token transfer event data from Ethereum and Polygon. Ethereum or Polygon data can be accessed directly through a node or an application programming interface (API) provider like Infura or Etherscan. The absence of account indexing in Ethereum or Polygon poses a challenge for retrieving all past transactions and token transfers of a specific address, as it requires scanning through all blocks and token transfer events emitted by designated token contracts. Fortunately, Etherscan offers an API Endpoint Module for ``Accounts`` that facilitates the retrieval of transactions and token transfer events for a given address. By using our network-specific address subsets, we were able to significantly decrease the number of required API calls. 
We gathered all transactions and token transfers up until block number \texttt{17,670,000} on Ethereum (July 11, 2023) and block number \texttt{44,990,000} on Polygon (July 12, 2023). In total, we collected more than 30 million %30,689,978
token transfers and more than 16 million %16,092,531
 normal transactions. The output was stored in CSV files and later imported  into a MongoDB\footnote{\url{https://www.mongodb.com/}} database. 

The database contains separate collections for transactions and transfers. Both collections share fields including the transaction hash, timestamp in seconds from the UNIX epoch, the amount of gas used in the transaction, the gas price specified by the user and the nonce (for EOAs, the nonce corresponds to the number of transactions sent beginning from 0).

The transaction data field includes a from and to address ('to' address either smart contract or EOA for simple transfers), a value field (native asset), the transaction's input (mostly referred to as calldata), the name of the function called (added by Etherscan, none if simple transfer of Ether or Matic). We added the chain Name to combine the data from both chains.

Token transfer receipts include the transaction hash, UNIX timestamp, sender and recipient address, nonce, gas price, the amount of gas, chain information, the token contract address and name, tokenType, and token standard specific fields: value, tokenDecimal for ERC-20, tokenID for ERC-721, and value/tokenID for ERC-1155.


\iffalse
Transactions: hash, from, to, timestamp, nonce, value, gasprice, gasUsed, input, functionName, chainName
Transfers: hash, from (sender), to (recipient), contractAddress (tokenContract address), value (amount, only for ERC20, ERC1155), nonce, tokenName, gas price, gas used, chainName, tokenType (token standard), token ID (only for ERC721/ERC1155), ((isSet, userAddress))

Transfer Events: 30,689,978
Transfer Events Ethereum: 7,832,778
Transfer Events Polygon: 22,857,200
Transactions = 16,092,531
Transactions Ethereum = 8,448,584
Transactions Polygon = 7,643,947

Most of the activity revolves around NFTs, especially on Polygon.
Figure X visualizes the number of daily transactions and token transfers for each chain.
\begin{figure}[h!]
	\centering
	\includegraphics[width=\textwidth]{../figures/transfers_tx_by_chain.png}
	\caption{Monthly Transactions and Token Transfers by chain}
	\label{fig:Data}
\end{figure} 
Transfer Events, adding Information (isInSet) \\
Transactions \\
Filtering, Intra-set transfers\\
Data Structure, Fields \\
We reduced this set to addresses that were active/recorded on either Ethereum or Polygon using Blockscan\footnote{\url{https://blockscan.com/}}. This reduced the address set from to X.
Using Blockscan\footnote{\url{https://blockscan.com/}}, we were able to reduce the address set to addresses that were recorded on either Ethereum or Polygon.
\fi


\subsection{Intra-set asset transfers}
Our goal is to cluster addresses belonging to the same real-world entity that were recorded in Decentraland during this 9-month period. Therefore, we disregard addresses outside of this set. 

Since all of our addresses are EOAs (some multisig smart contract wallets). -> Check this?

implications
advantage that we do not need to check if a specific address is a contract.

Filter transactions and transfers events where both the \texttt{from} (sender) and the \texttt{to} (recipient) address are in the address set

call/define it intra-set transfers: includes all token transfers and standard native asset transfers. Use this to generate network graph. or for our other clustering heuristics.

\subsection{ENS Names}
In order to evaluate some of our clustering methods, we need a ground truth. Similarly to \cite{Beres2020}, we use Ethereum Name Service identifiers as ground truth information.

ENS is a distributed, open, and extensible naming system based on the Ethereum blockchain. In spirit, it is similar to the well-known Domain Name Service (DNS). However, in ENS the registry is implemented in Ethereum smart contracts. ENS maps human-readable names like alice.eth to machine-readable identifiers such as Ethereum addresses. Therefore, ENS provides a more user-friendly way of transferring assets on Ethereum, where users can use ENS names as recipient addresses instead of the error-prone hexadecimal Ethereum addresses.

%When ENS .eth registrar migrated in May 2019, the .eth registrar became an 

%Currently, all the subdomains or non .eth domains are not NFT, unless the domain registrar itself supports NFT such as (dcl.eth, and .kred). If you want to turn all subdomains which you own, you have to create a registrar


Find all addresses that transferred/minted an ENS Name.

Check for each address in the list if the address points to an human-readable name.

If two addresses point to the same ENS name we consider them belonging to the same entity. 

We use ENS names with exactly two unique addresses to measure the performance of different profiling techniques.

assume ENS names that point to multiple addresses, see 


\begin{enumerate}
	\item From transfer event data, get addresses that interacted with ENS or DCL Registrar as a list
	\item For each address, find the ENS name it points to using web3.py.ens module (Infura as provider) w3.ens('0xAB') - Ens List
	\item Find addresses that point to the same human-readable name (group by ENS Name and find instances where exactly 2 addresses point to the same name)
	\item Save in a CSV (ens\_pairs.csv)
\end{enumerate}
 
%For detecting ground truth pairs in order to evaluate our node embedding methods, we gathered data from ENS contract...
% Check EDNS

From the web3.py documentation \url{https://web3py.readthedocs.io/en/stable/ens_overview.html}
%The Ethereum Name Service (ENS) is analogous to the Domain Name Service. It enables users and developers to use human-friendly names in place of error-prone hexadecimal addresses, content hashes, and more.
%The ens module is included with web3.py. It provides an interface to look up domains and addresses, add resolver records, or get and set metadata.



%%%%%%%%%%%%%%%%%%%%%%%%%%%%%
%%% Clustering Heuristics %%%
%%%%%%%%%%%%%%%%%%%%%%%%%%%%%

\section{Clustering Heuristics}
In this section, we provide a overview of clustering heuristics in the context of account-based blockchains and on-chain data. We will give a detailed explanation of the intuition/rationale behind the heuristic, the workings and discuss the applicability for our case. In Section 5 we will then apply and, if possible, evaluate the heuristics deemed suited for us.

%detailed explanation
% Goal: explain each heuristic, categorize, and discuss applicability to our address set

\subsection{Self-authorization}
The \texttt{approve} function required in all three token standards allows another address to spend tokens on behalf of the actual owners. This functionality is mainly used in connection with smart contracts, but can also be applied for regular EOAs. Users could approve another address they own, a reason for this might be risk distribution over several addresses with partial accessibility. \citep{FV:17}

\subsection{Deposit address reuse}
This heuristic exploits the common practice of crypto exchanges creating so-called deposit addresses for each user, which forward funds to a main address. Multiple addresses that send funds to the same deposit address are highly likely controlled by the same entity. \citep{FV:17}

\subsection{Airdrop multi participation}
Airdrops are popular token distribution mechanisms, and recipients are mostly chosen based on past protocol activity. A famous example is Uniswap's UNI airdrop, where each address that interacted with the protocol received a fixed amount of UNI tokens\footnote{\url{https://blog.uniswap.org/uni}}. These distribution mechanisms are often not Sybil-resistant, leading to people creating multiple addresses in anticipation of an airdrop. However, managing the tokens on all of these addresses is impractical, which is why they are often aggregated into one address. \citep{FV:17}

\subsection{Graph-based network analysis}
Token networks -> Victor, Casale Brunet

%The set of addresses used in interactions characterize a users. Users with multiple accounts might interact with the same addresses from most of them. Furthermore, as users move funds between their personal addresses, they may unintetionally reveal their address clusters. Clustering experiments conducted on a transaction/transfer graph with nodes as addresses and edges as transactions/transfers. Rozemberczki provides a library of node embedding methods to discover address pairs that might belong to the same user. Preprocessing steps: transfers as undirected edges, remove loops and multi-edges, exlcude nodes outside the largest connected component. Resulting graph. Applied 3 node embedding methods to this graph (Diff2Vec, Role2Vec and deep walk.
\textbf{Node embeddings}

Projects addresses to points in a low-dimensional vector space based on who it interacts with. In this vector space, adresses belonging to the same entity should be close together in Euclidian distance.

Consider constructing an undirected graph from all Ethereum transactions where nodes are composed of distinct addresses, and an edge is placed between two nodes if there is a transaction between them.

At this abstraction layer, we seek to learn a ``node embedding function``that projects a node to a d-dimensional vector representation. Importantly I want this embedding to capture semantic information about the node, such as which other addresses it frequently interact with. To do so, we leverage popular graph representation learning algorithms. (Grover and Leskovec, 2016; Rozemberczki and Sarkar, 2018). In particular, we focus on Diff2Vec, Role2Vec and 


Given an address, to find its cluster, we can search for the closest k vectors in Rd. In practice, we accomplish this efficiently using FAISS (Johnson et al., 2019). This will always return k addresses.

Node embedding methods form a class of network representation learning methods that map graph nodes to vectors in a low-dimensional space. They are designed to represent vertices with similar neighborhood structure by vectors that are close in the vector space. Intuitively, addresses that interact with the same set of addresses in the Ethereum transaction graph should be close in the embedded space. Research in node embedding has been catalyzed by Word2Vec, an embedding method for natural language processing. 
... In this work, we use these techniques on the Ethereum transaction graph to link addresses owned by the same user.
...
The set of addresses used in interactions characterize a user. Users with multiple accounts might interact with the same addresses or services from most of the. Furthermore, they may unintentionally reveal their address clusters. Our deanonymization methods are conducted on a transaction graph with nodes as Ethereum addresses and edges as transactions. From the library of \cite
Diff2Vec, Role2Vec, DeepWalk \\
Transaction Graph Analysis

\subsection{Time-of-day transaction activity}
Transaction timestamps reveal the daily activity patterns of the account owner -> example
Given the set of timestamps, an account is represented by the vector including the mean, median and standard deviation, as well as the time-of-day activity histogram dividid into $b_{hour}$ bins. We chose 50 bins (from beres)

\subsection{Gas price distribution}
Gas price definition, often set by the wallet software (slow, medium, fast), given the changes in traffic volume we normalice the gas price bxy the daily network average in our dataset. Make example. Given the normalized gas prices of the transactions sent, an account is represented by the mean, median and standard deviation, as well as the normalized gas price histogram divided into $b_{gas}$ bins. We chose 6 bins (4 hour intervals) according to Beres.



%%%%%%%%%%%%%%%%%%%%%
%%% Data Analysis %%%
%%%%%%%%%%%%%%%%%%%%%

\section{Data Analysis}




%%%%%%%%%%%%%%%%%%
%%% Discussion %%%
%%%%%%%%%%%%%%%%%%

\section{Discussion}



%%%%%%%%%%%%%%%%%%
%%% Conclusion %%%
%%%%%%%%%%%%%%%%%%

\section{Conclusion}



%%%%%%%%%%%%%%%%%%%%%%%%%%%%
%%% Literaturverzeichnis %%%
%%%%%%%%%%%%%%%%%%%%%%%%%%%%

\newpage
\setcounter{page}{1}
\pagenumbering{roman}
\onehalfspacing
\addcontentsline{toc}{section}{References}
\bibliography{mybib}
\bibliographystyle{agsm}

%\section{Appendix}
\end{document}


%%% Table
%\begin{table}[h!]
%  \center
%  \begin{tabular}{lcc}
%    \hline\hline
%    Header & Header & Header \\ \hline
%    Entry 1 & $0 \leq x<1$ & $\alpha$\\
%    Entry 2 & $x=1$ & $\beta$\\
%    Entry 3 & $x>1$ & $\gamma$\\
%    \hline\hline
%  \end{tabular}
%  \caption{This is a table}
%  \label{tbl:test}
%\end{table}
%%%
