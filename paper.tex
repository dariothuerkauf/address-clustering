%%%%%%%%%%%%%%%%%%%%%%%%%%%%%%%%%%%%%%%%%%%%%%
%%%%%%%%%%%%%%%%%%%%%%%%%%%%%%%%%%%%%%%%%%%%%%
%%% Master Thesis Template by Fabian Schär %%%
%%%%%%%%%%%%%%%%%%%%%%%%%%%%%%%%%%%%%%%%%%%%%%
%%%%%%%%%%%%%%%%%%%%%%%%%%%%%%%%%%%%%%%%%%%%%%

%%%%%%%%%%%%%%%%%%%%%%%%%%%%%%%%%%%%%%
%%% Packages and Document Settings %%%
%%%%%%%%%%%%%%%%%%%%%%%%%%%%%%%%%%%%%%

\documentclass[12pt,a4paper,titlepage,oneside,english]{article}

%%% Main Packages %%%
\usepackage[english]{babel}
%\usepackage[ngerman]{babel} % Use this option for German settings.
\usepackage[T1]{fontenc}
\usepackage[utf8]{inputenc}

%%% Additional Packages %%%
\usepackage{cite}
\usepackage{framed}
\usepackage{graphicx}
%\usepackage[german]{fancyref}
\usepackage[german,hidelinks]{hyperref} %hidelinks
\usepackage{multirow}
\usepackage[round]{natbib}
\usepackage{setspace}
\usepackage{geometry}
\usepackage{pst-all} % Not working with Sweave!!!

%%% Math Packages %%%
\usepackage{amsmath}
\usepackage{amstext}
\usepackage{amssymb}
\usepackage{theorem}
\usepackage{epsfig}
\usepackage{longtable}

%%% Layout Specifications %%%
\geometry{a4paper, top=35mm, left=40mm, right=40mm, bottom=45mm,
headsep=10mm, footskip=12mm}

%%% Parskip Settings %%%
\setlength{\parskip}{3mm}
\setlength{\parindent}{0mm}

%%% Document Specifications %%%
\title{Template Master's Thesis}
\author{John Doe}


%%%%%%%%%%%%%%%%%%
%%% Title Page %%%
%%%%%%%%%%%%%%%%%%

\begin{document}
%\begin{titlepage}
\begin{center}
\vspace{1em}
\large{Master Thesis}\\
\huge Developing Address Clustering Heuristics for Account-Based Blockchain Networks:\\ An Analysis based on a Specific Address Set \\
\Large \vspace{1em}
Dario Thürkauf
\end{center}

\vspace{1em}
\normalsize
\begin{flushleft}
Supervised by:\\ 
Prof. Dr. Fabian Schär \\
Credit Suisse Asset Management (Schweiz) Professor for \\ 
Distributed Ledger Technologies and Fintech \\
Center for Innovative Finance, University of Basel
\end{flushleft}

\vspace{1em}
\onehalfspacing
\begin{center}
\section*{Abstract}
\end{center}
Assentiar consuetae ha opinionum mentemque ob ii. Ne conflantur de intelligat et me cohibendam. Imaginandi ob to at agnoscerem et mutationum. In methodum ob ii at quicquid lectorum. Procuravi ha dependent ob evidenter tangantur concipere. Immortalem objectivus deo eae rei attingebam ita advertebam quamprimum. Typis patet prius qua nia mem ens. Suppono sim ita pendere nam agnosci quopiam vestiri spondeo dum. Tes illum mundo vetus signa fit talem res his.  \\
\vfill
\textbf{Keywords:} Keyword 1, Keyword 2, Keyword 3, Keyword 4.\\
\noindent\textbf{JEL:} X00, X00, X00
%\end{titlepage}


%%%%%%%%%%%%%%%%%%%%%%%%%%%%%
%%% Contents & Declaration%%%
%%%%%%%%%%%%%%%%%%%%%%%%%%%%%

\pagenumbering{gobble}

\newpage
\pagenumbering{Roman}
\tableofcontents

\vfill
\begin{center}
%\includegraphics[width=4cm]{../figures/logo_cif.png}
\end{center}
\newpage
\singlespacing
%\vspace{-1.5cm}
\section*{Declaration of Independent Authorship}
I attest with my signature that I have written this work independently and without outside help. I also attest that the information concerning the sources used in this work is true and complete in every respect. All sources that have been quoted or paraphrased have been marked accordingly. 
Additionally, I affirm that any text passages written with the help of AI-supported technology are marked as such, including a reference to the AI-supported program used. This paper may be checked for plagiarism and use of AI-supported technology using the appropriate software. I understand that unethical conduct may lead to a grade of 1 or ``fail`` or expulsion from the study program.\\

Dario Thürkauf

\begin{figure}[h!]
	\centering
	\hspace{-10cm}
	%\includegraphics[width=3cm]{../figures/signature.jpeg}
\end{figure}
%%%%%%%%%%%%%%%%%%%%

\newpage
\onehalfspacing
\pagenumbering{arabic}


%%%%%%%%%%%%%%%%%%%
%%% Introduction%%%
%%%%%%%%%%%%%%%%%%%

\section{Introduction}

Public permissionless blockchains such as Bitcoin \citep{nakamotoBitcoin2008} and Ethereum \citep{buterin2014ethereum} allow users %individuals
 to participate with multiple pseudonymous addresses. The creation of these addresses is virtually cost free. Contrary to popular belief, these blockchains are entirely transparent. All transactions are stored as part of the blockchain's history and publicly observable.
This has opened up a nascent scientific field dealing with entity identification within blockchain networks. Researchers try to cluster addresses controlled by the same user by analyzing on-chain data and detecting usage patterns. The majority of these methods are heuristic which often makes them difficult to evaluate due to the absence of any ground truth. \newline
On one hand, identifying the addresses that belong to the same real-world entity is beneficial. According to \cite{FV:17} it allows better evaluation of network properties with respect to usage, distribution of wealth and detecting fraudulent activities. For instance, if a user distributes their voting power across various addresses, they might manipulate an on-chain voting process that seems fair at the outset. \newline
Conversely, the lack of privacy is also detrimental to most financial use cases (DeFi). As a response, multiple privacy-enhancing tools and protocols have been proposed to obfuscate transaction tracing. 
Nonetheless, these protocols are not yet widely adopted and careless user behaviour can undermine the privacy guarantees they offer. \newline
%As such/In conclusion
The topic of privacy, or lack thereof, will remain significant for future blockchain development and research.

%However, they are not widely used even cannot guarantee perfect privacy (factors: size of anonymity set, usage /user behaviour of protocol, privacy guarantees depend on other users).
%The most prominent/widely used non-custodial mixer is Tornado cash (Schär, Tutela).
%This reaction has, in turn, sparked the development of other clustering heuristics exploiting user behavior interacting with privacy-enhancing protocols. 

\textbf{Related work}\\
Previous work in this field can be broadly distinguished between heuristics developed for the unspent transaction output (UTXO) model (e.g Bitcoin, ZCash) and the account-based model (e.g. Ethereum, Polygon). While both models share the concept of addresses, the notion of accounts is not present in UTXO-based blockchains. The way in which transactions are processed is fundamentally different, and therefore, clustering heuristics are not applicable to both paradigms.
%\textbf{UTXO-model clustering heuristics} 
A number of address clustering heuristics in the UTXO setup have been proposed for Bitcoin and derivatives \citep{Androulaki2013, Meiklejohn2013, Haslhofer2016, jourdan2018, kappos2022}. All these methods will not be the focus of this work and therefore not discussed. As suggested by Nakamoto in the Bitcoin whitepaper, most Bitcoin wallet implementations use a new key pair for each transaction to keep them from being linked to a common owner. \newline
In contrast to the UTXO-model, native transactions in account-based blockchains can only move funds between a single sender and receiver, and the ``change``remains in the sender account. Subsequent transactions necessarily use the same address again. The account-based model essentially relies on address reuse on the protocol level. Therefore, privacy guarantees should be lower and most users likely use only a small number of addresses. \citep{Beres2020} \newline
% Make an assumption that people only use a small set of addressesyy
%\textbf{Account-model clustering heuristics}
Clustering heuristics for account-based blockchains were first introduced by \cite{FV:17}. He proposed and applied heuristics that exploit patterns related to deposit addresses, multiple participation in airdrops, and token authorization mechanisms. \newline
\cite{Beres2020} propose more universally applicable methods, as they argue that Victor’s heuristics, while powerful, assume participation in certain on-chain events. Their technique interprets transactions or token transfers as network graphs, with addresses as nodes and asset transfers as edges. They quantitavely compare graph-representation learning algorithms (a subset of machine learning) and propose further user profiling techniques based on time-of-day activity and transaction fees. Using ENS address pairs as ground truth (quasi-identifiers), the authors rigorously test their methods and apply their findings to significantly reduce the privacy guarantees of Tornado Cash. \newline
\cite{wu2022tutela} extend on one of \cite{Beres2020} graph representation learning (node-embedding) algorithms and apply it at scale. Further, they also propose a set of new clustering heuristics targeting Tornado Cash, showing that careless user behavior can still reveal identity. They build an application built on those heuristics to measure the anonymity of an Ethereum address. \newline
All of the methods mentioned will be discussed in greater detail in Section 3. \newline
Broadening the scope of entity identification, \cite{victorlüders2019} analyzed the largest Ethereum ERC-20 token networks from a graph perspective. Similarly, \cite{casalebrunet2021} apply a similar analysis to various non-fungible token (NFT) ecosystems. Both find similarities to interaction graphs in social networks (hub-and-spoke). Further, \cite{Payette2017} propose a segmentation of Ethereum address space into four distinct behaviour groups sharing similar attributes using k-means clustering. 

\cite{yu2023} propose a novel approach for correlation analysis by exploiting network information, rather than treating users as entities and analyzing on-chain data. Although this approach has the potential to avoid the impact of privacy-enhancing technologies, it introduces a new set of limitations and problems. Thus, this approach may be of great interest for further research, particularly when privacy-enhancing techniques become more widely used.


\textbf{Our contribution}\\
Due to Decentraland's blockchain-based architecture, each avatar contains information about a user's Web3 address. 
To obtain these addresses, a separate project collected snapshots of avatar activity over an ten-month period from July 2022 to April 2023. 
In this work, we perform entity identification on this address set containing 473,927 addresses. 
The applicability of existing heuristics is tested, and if feasible, their efficacy in detecting entities using multiple addresses is evaluated. 
The objective is to estimate the number of real-world entities that are represented by these addresses. Our focus is on clustering addresses within this set alone, while disregarding possible clusters outside of it that were not active in Decentraland during the given timeframe. 
Furthermore, we introduce our own heuristics for identifying address clusters and evaluate their suitability for implementation in other contexts.
%Anwenden/discuss and evaluate existing heuristics to my specific address set.
%Anpassen, erweitern
%Eigene, spezifische Heuristiken ausdenken

The remainder of this paper is structured as follows: In Section 2, we provide a brief overview of the key concepts (basics, necessary for understanding the setup), including Ethereum accounts and the EVM, tokens, Decentraland and privacy-enhancing protocols. In Section 3, we describe our data collection methodology and the data preparation. Section 4  describes different clustering heuristics in detail. Section 5 applies the clustering heuristics from 4. Finally we discuss the results of our analysis and summarize our findings in Sections 5 and 6.

% Motivation \& Relevance of the topic \\
% Transaction data publicly available
% Not really good to hide illicit business acitivities.
% First, this was exploited in Bitcoin and Bitcoins UTXO model
% Recently also account-based
%Literarure Review: Still nascent field, First start with bitcoin/UTXO based, go over to account-based: Victor, Beres et al., Tutela \\



%%%%%%%%%%%%%%%%%%%%%
%%% Preliminaries %%%
%%%%%%%%%%%%%%%%%%%%%

\section{Preliminaries}

\textbf{Contract Accounts and EOAs}\\
Account-based blockchains usually distinguish between \textit{externally owned accounts} (EOAs) and \textit{contract accounts} (smart contracts). EOAs are created and controlled by private keys and can be used to hold the native protocol asset, send and receive transactions, and interact with contract accounts. Contract accounts are controlled by the contract's code, their state can be modified through transactions sent to the contract and they cannot initiate transactions. \citep{buterin2014ethereum} \newline Each account has a 20-byte address encoded in hexadecimal. For EOAs, this address is based on the last 20 bytes of the Keccak-256 hash of the ECDSA public key. For contract accounts, it is defined as the last 20 bytes of the Keccak-256 hash of the RLP encoding of the sender address and account nonce. \citep{GW:14} \newline
In addition, \cite{eip1014} introduces a new contract creation mechanism allowing for more predictable contract addresses.%EIP-1014

Ethereum Mainnet and Polygon PoS are both account-model blockchains and use the same execution logic, the Ethereum Virtual machine. 


- Polygon PoS is a so-called sidechain, closely associated with Ethereum mainnet and used for scaling

%\textbf{Ethereum, Polygon, EVM}\\


\textbf{Transactions and Transfer Events}\\

\textbf{Tokens and Token Standards}\\
%shorter than in Seminar Thesis, comparable to Metaverse Retailing Paper

\textbf{Metaverse and Decentraland} \\
- Much of the Decentraland infrastructure has been deployed on Ethereum Mainnet

\textbf{Privacy-enhancing protocols} \\



%%%%%%%%%%%%
%%% Data %%%
%%%%%%%%%%%%

\section{Data Collection and Preparation}

The original dataset comprised 473,927 distinct (Web3) addresses. As we only focus on clustering methods using on-chain transaction or token transfer data, we exclude addresses that have not been recorded on either Ethereum or Polygon. To accomplish this, we used data from Blockscan\footnote{\url{https://blockscan.com/}}. Overall, 59,651 addresses were recorded on Ethereum, 129,988 on Polygon, with 52,095 addresses appearing on both networks simultaneously. The set of remaining addresses consists of 137,544 addresses. Subsets are visualized in a Venn diagram in Figure 1. 
These addresses were used to collect transactions and token transfer events from Ethereum Mainnet and Polygon PoS. Ethereum or Polygon data can be accessed directly through a node or an application programming interface (API) provider like Infura or Etherscan. The absence of account indexing in Ethereum or Polygon poses a challenge for retrieving all past transactions and token transfers of a specific address, as it requires scanning through all blocks and token transfer events emitted by designated token contracts. Fortunately, Etherscan offers/provides an API Endpoint Module for ``Accounts`` that facilitates the retrieval of transactions and token transfer events for a given address. By using our network-specific address subsets, we were able to significantly decrease the number of required API calls. 
We gathered all transaction and token transfer data of the addresses up until block 17,670,000 on Ethereum (July 11, 2023) and block 44,990,000 on Polygon (July 12, 2023). The output was saved in comma-separated value (CSV) files and imported into a MongoDB database\footnote{\url{https://www.mongodb.com/}}.

Statistics about data, how many transactions, how many transfers, size
Transfer events = 30,689,978
Transactions = 16,184,993, how many polygon/ethereum, table

\begin{table}[h]
\centering
\begin{tabular}{lcc}
\hline \hline
  & Transactions & Transfers \\
 \hline
 Polygon & Cell 2 & Cell 3 \\
 Ethereum & Cell 5 & Cell 6 \\
 \hline
 Total & 16,184,993 & 30,689,978 \\
 \hline \hline
\end{tabular}
\caption{Size of raw data}
\label{tab:data_size}
\end{table}


The transaction data fields contain the transaction hash, from and to address, the timestamp in seconds from the UNIX epoch, a nonce, value, input (often reffered to as calldata), the amount of gas used in the transaction, the specified gas price, the name of the function called (added by Etherscan) and the name of the blockchain.
Token transfer receipts include the transactions hash, timestamp, from and to address, nonce, gas price, the amount of gas used, chain information, the token contract address and name, tokenType, and token standard specific fields: value, tokenDecimal for ERC-20, tokenID for ERC-721, and value/tokenID for ERC-1155.


%Transfer Events, adding Information (isInSet) \\
%Transactions \\
%Filtering, Intra-set transfers\\
%Data Structure, Fields \\
%We reduced this set to addresses that were active/recorded on either Ethereum or Polygon using Blockscan\footnote{\url{https://blockscan.com/}}. This reduced the address set from to X.
%Using Blockscan\footnote{\url{https://blockscan.com/}}, we were able to reduce the address set to addresses that were recorded on either Ethereum or Polygon.

For detecting ground truth pairs in order to evaluate our node embedding methods, we gathered data from ENS contract...
Additional data, ENS \\


%%%%%%%%%%%%%%%%%%%%%%%%%%%%%
%%% Clustering Heuristics %%%
%%%%%%%%%%%%%%%%%%%%%%%%%%%%%

\section{Clustering Heuristics}
% Goal: explain each heuristic, categorize, and discuss applicability to our address set

Self-authorization \\
Deposit address reuse \\
Airdrop multi participation \\

Graph-based network analysis / Transaction Graph Analysis
Token networks -> Victor, Casale Brunet

%The set of addresses used in interactions characterize a users. Users with multiple accounts might interact with the same addresses from most of them. Furthermore, as users move funds between their personal addresses, they may unintetionally reveal their address clusters. Clustering experiments conducted on a transaction/transfer graph with nodes as addresses and edges as transactions/transfers. Rozemberczki provides a library of node embedding methods to discover address pairs that might belong to the same user. Preprocessing steps: transfers as undirected edges, remove loops and multi-edges, exlcude nodes outside the largest connected component. Resulting graph. Applied 3 node embedding methods to this graph (Diff2Vec, Role2Vec and deep walk.

Diff2Vec, Role2Vec, DeepWalk \\
Time-of-day transaction activity \\
Transaction timestamps reveal the daily activity patterns of the account owner -> example
Given the set of timestamps, an account is represented by the vector including the mean, median and standard deviation, as well as the time-of-day activity histogram dividid into $b_{hour}$ bins. We chose 50 bins (from beres)
Gas price distribution \\
Gas price definition, often set by the wallet software (slow, medium, fast), given the changes in traffic volume we normalice the gas price bxy the daily network average in our dataset. Make example. Given the normalized gas prices of the transactions sent, an account is represented by the mean, median and standard deviation, as well as the normalized gas price histogram divided into $b_{gas}$ bins. We chose 6 bins (4 hour intervals) according to Beres.



%%%%%%%%%%%%%%%%%%%%%
%%% Data Analysis %%%
%%%%%%%%%%%%%%%%%%%%%

\section{Data Analysis}




%%%%%%%%%%%%%%%%%%
%%% Discussion %%%
%%%%%%%%%%%%%%%%%%

\section{Discussion}



%%%%%%%%%%%%%%%%%%
%%% Conclusion %%%
%%%%%%%%%%%%%%%%%%

\section{Conclusion}



%%%%%%%%%%%%%%%%%%%%%%%%%%%%
%%% Literaturverzeichnis %%%
%%%%%%%%%%%%%%%%%%%%%%%%%%%%

\newpage
\setcounter{page}{1}
\pagenumbering{roman}
\onehalfspacing
\addcontentsline{toc}{section}{References}
\bibliography{mybib}
\bibliographystyle{agsm}

%\section{Appendix}
\end{document}


%%% Table
%\begin{table}[h!]
%  \center
%  \begin{tabular}{lcc}
%    \hline\hline
%    Header & Header & Header \\ \hline
%    Entry 1 & $0 \leq x<1$ & $\alpha$\\
%    Entry 2 & $x=1$ & $\beta$\\
%    Entry 3 & $x>1$ & $\gamma$\\
%    \hline\hline
%  \end{tabular}
%  \caption{This is a table}
%  \label{tbl:test}
%\end{table}
%%%
